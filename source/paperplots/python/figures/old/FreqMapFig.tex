\documentclass[10pt,english]{article}
\usepackage[T1]{fontenc}
\usepackage[latin1]{inputenc}
\usepackage{pslatex}
\usepackage{lscape}
\setlength\parskip{\medskipamount}
\setlength\parindent{0pt}
\usepackage{a4wide}
\usepackage{array}	
\usepackage{amssymb}

\usepackage{babel}
\usepackage{multirow}
\usepackage[margin=1mm]{geometry}
\usepackage[usenames]{color}
\definecolor{RoyalPurple}{cmyk}{0.75,0.9,0,0.1}
\definecolor{MyBlue}{rgb}{0.0025,0.1125,0.2}

\usepackage[pdftex]{graphicx}
\usepackage[pdftex]{graphics}
\usepackage{longtable}
\usepackage[pdftex, colorlinks, backref=page, pdfhighlight=/P,pdfpagelayout=OneColumn,
	plainpages=false, breaklinks, naturalnames=true, linkcolor=MyBlue, urlcolor=blue, citecolor=RoyalPurple]{hyperref}	%	hypertexnames=true
\renewcommand{\topfraction}{.001}
\renewcommand{\bottomfraction}{.001}
\renewcommand{\textfraction}{.00015}
\renewcommand{\floatpagefraction}{.99}
\renewcommand{\dbltopfraction}{.99}
\renewcommand{\dblfloatpagefraction}{.99}

\begin{document}

\title{HFI - LFI Frequency Maps}

%%%%%%%%%%%%%%%%%%%%%%%%%%%%%%%%%%%%%%%%%%%%%%%%%%%%%%%%%%%%%%%%%%%%%%

% This is just an example template, there may be better or more efficient ways to do this.
% Change the filenames to those you want, adjust the number of plots to what you need.
% The trim, and clip arguments remove the individual colorbars.  This is on a 180 m wide figure scaled down to 88mm wide; it may need some tweaking for other sizes.
% The colourbars are left at the original size, with a different one for 545 and 857 as they have different units.
% This code template could be inserted within an \begin{figure] \end{figure} environment within your paper, 
% or the geometry package could be used to have the new figure beome a single eps or pdf file.

\begin{center}
\begin{tabular}{cc}
\includegraphics[trim= 0 60 0 0, clip, width=88mm]{PlanckFig_map_universal_python_180mm_30GHz} &
\includegraphics[trim= 0 60 0 0, clip, width=88mm]{PlanckFig_map_universal_python_180mm_44GHz} \\
\includegraphics[trim= 0 60 0 0, clip, width=88mm]{PlanckFig_map_universal_python_180mm_70GHz} & 
\includegraphics[trim= 0 60 0 0, clip, width=88mm]{PlanckFig_map_universal_python_180mm_100GHz} \\
\includegraphics[trim= 0 60 0 0, clip, width=88mm]{PlanckFig_map_universal_python_180mm_143GHz} & 
\includegraphics[trim= 0 60 0 0, clip, width=88mm]{PlanckFig_map_universal_python_180mm_217GHz} \\
\multicolumn{2}{c}{\includegraphics[trim= 0 60 0 0, clip, width=88mm]{PlanckFig_map_universal_python_180mm_353GHz}} \\
\multicolumn{2}{c}{\includegraphics[trim= 0 0 0 200, clip, width=180mm]{PlanckFig_map_universal_python_180mm_353GHz}} \\
\includegraphics[trim= 0 60 0 0, clip, width=88mm]{PlanckFig_map_universal_python_180mm_545GHz} & 
\includegraphics[trim= 0 60 0 0, clip, width=88mm]{PlanckFig_map_universal_python_180mm_857GHz} \\
\multicolumn{2}{c}{\includegraphics[trim= 0 0 0 200, clip, width=180mm]{PlanckFig_map_universal_python_180mm_857GHz}} % \\
\end{tabular}
\end{center}

\end{document}
